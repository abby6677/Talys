\begin{samplecase}
{\bf Recoils: 20 MeV  n + ${}^{28}$Si}\newline
In this sample case, we calculate the recoils of the residual nuclides
produced by 20 MeV neutrons incident on ${}^{28}$Si reaction. Two methods are
compared.
\subsubsection{Case a: ``Exact'' approach}
In the exact approach, each excitation energy bin of the population of each
residual nucleus is described by a full distribution of kinetic recoil
energies.
The following input file is used

\VerbatimInput{\samples n-Si028-recoil-exact/org/talys.inp}

{\small \begin{verbatim}

#
# General
#
projectile n
element si
mass 28
energy 20.
#
# Parameters
#
m2constant 0.70
sysreaction p d t h a
spherical y
#
# Output
#
recoil y
filerecoil y
\end{verbatim} } \renewcommand{\baselinestretch}{1.07}\small\normalsize
\noindent
For increasing incident energies, this calculation becomes quickly
time-expensive.
The recoil calculation yields separate files with the recoil spectrum per
residual nucleus, starting with {\it rec}, followed by the Z, A and incident
energy, e.g. {\em rec012024spec020.000.tot}. Also following additional
output block is printed:

{\small \begin{verbatim}

 8. Recoil spectra

 Recoil Spectrum for  29Si

  Energy   Cross section

   0.018 0.00000E+00
   0.053 0.00000E+00
   0.088 0.00000E+00
.....................................
   0.676 3.21368E+00
   0.720 3.22112E+00
   0.763 3.21215E+00
   0.807 3.10619E+00
   0.851 2.60156E+00
   0.894 2.30919E-01

 Integrated recoil spectrum       :  1.04358E+00
 Residual production cross section:  9.71110E-01
\end{verbatim} } \renewcommand{\baselinestretch}{1.07}\small\normalsize


\subsubsection{Case b: Approximative approach}
As an approximation, each excitation energy bin of the population of each
residual nucleus is described by a an average kinetic recoil energy.
For this, we add one line to the input file above,
{\small \begin{verbatim}

recoilaverage y
\end{verbatim} } \renewcommand{\baselinestretch}{1.07}\small\normalsize
\noindent
The results, together with those of case (a), are compared in
Fig.~\ref{CM2LAB}.
\end{samplecase}
