\begin{samplecase}
{\bf Calculations up to 500 MeV for p + ${}^{209}$Bi}\newline
In TALYS, the maximum allowed incident energy has formally been extended to 1 GeV. 
The term 'formally' means that the code seems to produce reasonable results, without 
crashing, while we are well aware that the physics beyond about 200 MeV is different 
from that at lower energies. The question is at which energy 
the ``low-energy'' models will 
start to fail. To investigate this, the extension to 1 GeV has been made.
In this sample case we perform calculations for p + ${}^{209}$Bi at 5 different 
energies,

\VerbatimInput{\samples p-Bi209-500MeV/org/talys.inp}

Here we use the option to give equidistant incident energies from 100 to 500 MeV with 
steps of 100 MeV.
All results can be found in the usual files, e.g. the residual production cross 
sections of $^{204}$Pb are in {\it rp082204.tot}, which look as follows
{\small \begin{verbatim}

# p + 209Bi: Production of 204Pb - Total
# Q-value    =-4.42737E+00 mass= 203.973043
# E-threshold= 4.44871E+00
# # energies =  5
#    E         xs
1.000E+02 2.81964E+01
2.000E+02 3.70364E+01
3.000E+02 4.09552E+01
4.000E+02 3.84491E+01
5.000E+02 3.62129E+01
\end{verbatim} } \renewcommand{\baselinestretch}{1.07}\small\normalsize
\noindent
Obviously, the user may extend this case up to 1 GeV and with a finer incident energy 
grid.
\end{samplecase}
